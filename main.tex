\documentclass{sm}
\usepackage[brazil]{babel}									%	Português do Brasil
\usepackage[utf8]{inputenc}									% Caracteres UTF-8 
\usepackage{amsmath,amsfonts,amssymb}				% Símbolos Matemáticos

\evento{III Semana do Matemático -- IME/UERJ}
\periodo{6 a 8 de maio de 2019}
\cabecalho{COMUNICAÇÃO CIENTÍFICA}

% MUDE A PARTIR DAQUI

\titulo{Título do Trabalho}
\autores{Autor-1\\Autor-2}
\curso{Curso}
\orientadores{Orientador-1\\Orientador-2}

\begin{document}

	\mostratitulo
	\mostradados															% Comente para formato cego
	
	\section{Resumo}
	
	Resumo, contendo de \textbf{200} a \textbf{500} palavras, com um único parágrafo, fonte Arial 12, justificado, espaço entre linhas 1.5 sem recuo e sem referências.
	
	\textbf{Palavras-chave:} separadas por vírgula.

	\vspace{\baselineskip}

	\textbf{Atenção:} As expressões matemáticas devem ser inseridas utilizando os comandos\\ \verb!\begin{align}! e \verb!\end{align}! do pacote amsmath.

	\vspace{\baselineskip}

	OBS: Nos enviar os arquivos \LaTeX \texttt{.tex} e \texttt{.pdf}.
	
\end{document}	
